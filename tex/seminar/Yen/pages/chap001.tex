%
% chap001.tex
%

\chapter{Einleitung}

%======================================================

%======================================================

Je mehr man sich in die Materie der Mathematik vertieft,
umso intensiver bedient man sich des reichen Vorrats an
abstrakten mathematischen Symbolen, um Gleichungen und
logische Aussagen kompakt und präzise
auszudrücken.
Möchte ein mathematisch versierter Mensch z.B. die
Gauß'sche Summenformel herleiten und nachweisen,
tendiert er meist dazu, dies durch trockene Gleichungen
und langwierige Induktionsschritte zu bewerkstelligen.
Das resultiert in eine oft für den Laien kompliziert
aussehende Beweisführung, die ihm schnell die Lust
auf schöne Mathematik verderben lässt.
Dabei kann die Mathematik auch mal bunt und attraktiv
im wahrsten Sinne des Wortes sein, wie sie in dieser
Seminarabeit dargestellt wird.

Diese Arbeit entstand im Rahmen des im Sommersemester
2021 unter der Leitung des Herrn Prof. Dr. Andreas Görg stattfindenden \textit{mathematischen Seminars}.
Wie auch der Titel erahnen lässt, werden
in dieser Arbeit
mit Hilfe der
\textit{unterschiedlich gefärbten Steinen}
mehrere Summenformeln
(Summe der ersten $n$ natürlichen Zahlen,
Summe der ersten $n$ ungeraden natürlichen Zahlen,
Quotienten von Summen ungerader natürlicher Zahlen
et cetera)
auf äußerst anschauliche, visuelle und einfache
Weise hergeleitet.
Diese bildhafte Methode der
\textit{Mustern aus bunten Steinen}
wurde ebenfalls seit dem 5. Jahrhundert v. Chr.
von den Pythagoreern angewandt, um verschiedene
mathematische Gesetzmäßigkeiten zu
veranschaulichen.
