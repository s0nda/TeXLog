%
% chap004.tex
%

\chapter{Quotienten von Summen ungerader natürlicher Zahlen}

Galileo Galilei (1564 - 1642) bemerkte, dass eine spezielle
Eigenschaft für gerade Anzahl aufeinanderfolgender ungerader
Zahlen erfüllt ist, nämlich:

\textit{
Das Verhältnis (Quotient) der Summe der ersten Hälfte der
ungeraden Zahlen zur Summe der verbleibenden
Hälfte der natürlichen Zahlen beträgt immer
$\frac{1}{3}$}.

\[
 \frac{1}{3} = \frac{1+3}{5+7} = \frac{1+3+5}{7+9+11}
 = \frac{1+3+5+7}{9+11+13+15} = \dots
\]

Aus diesem Quotienten erhält man die allgemeine Formel:
\[
 \frac{1}{3} = \frac{1+3+5+\dots+(2n-1)}{(2n+1)+(2n+3)+(2n+5)+\dots+(2n+(2n-1))}
\]

