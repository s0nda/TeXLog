%
% chap005.tex
%

\chapter{Darstellung einer natürlichen Zahl als Summe aufeinanderfolgender
natürlicher Zahlen}

Es stehen zahlreiche Methoden zur Darstellung einer natürlichen Zahl
als Summe aufeinanderfolgender natürlicher Zahlen zur Verfügung.
Hier wird ein Beispiel zur Darstellung der Zahl $70$ als Summe
aufeinanderfolgender natürlicher Zahlen angeführt.

Um die Zahl $70$ darzustellen, hat man drei Möglichkeiten.
Die erste Möglichkeit ist die Aufteilung der Zahl $70$ in
zwei Mal $35$. Diese führt zu einen Rechteck mit 2 Reihen
mit jeweils 35 Steinen.

\begin{figure}[H]
  \centering
  \includegraphics[width=.5\linewidth]{./images/muster05.png}
  \caption[]{Aufteilung der Zahl $70$ in zwei mal $35$.}
  \label{fig:muster_70_zweimal_35}
\end{figure}

$70$ ist eine gerade, durch 5 teilbare natürliche Zahl.
Man bildet $70$ durch $5 \cdot 14$.
Dadurch hat man ein Rechteck mit 5 Säulen, die jeweils
14 Steinen enthalten. Darüber hinaus gibt es eine
alternative Methode, bei der die Zahl $14$ um $1$ oder $2$
addiert oder subtrahiert wird, d.h.
$70 = 12 + 13 + 14 + 15 + 16$.
Also die Zahl $70$ lässt sich als Summe von 5 aufeinanderfolgenden
natürlichen Zahl darstellen.
