%
% settings.tex
%
%======================================================
% import packages
%======================================================
\usepackage[utf8]{inputenc}%\usepackage[T1]{fontenc}
\usepackage[ngerman]{babel}
\usepackage[autostyle]{csquotes}% load "csquotes" when working with "babel" and "bibtex" to avoid
                                % the warning "'babel/polyglossia' detected but 'csquotes' missing".
\usepackage{amsmath,amsthm,amssymb}
%\usepackage[sc]{mathpazo}% sc = small-cap
\usepackage{mathptmx}% supports Adobe Times Roman (or equivalent) as default text font.
\usepackage{bm}% display bold maths symbols.
\usepackage{float}% used with \begin{figure}[H] to set fixed position.
\usepackage{graphicx}
\usepackage{caption}
\usepackage{imakeidx}% create index for later looking-up
\usepackage{hyperref}
\usepackage{hypcap}
\usepackage[backend=biber,style=numeric,]{biblatex}% bibliography processor [biber] for BibLatex,
                                                   % [style=alphabetic,]


%======================================================
% configurations
%======================================================
\makeindex % create index for later looking-up.
\setlength\parindent{0pt}% set \noindent for entrie file.
%
% command for setting indentation and parskip in minipage.
\setlength{\parskip}{\medskipamount}% set vertical spacing between two paragraphs (glocal), \parskip = paragraph skip
\makeatletter 
\newcommand{\@minipagerestore}{%
  \setlength{\parindent}{0pt}% set indentation at paragraph begin (local)
  \setlength{\parskip}{\medskipamount}% set vertical spacing between two paragraphs (local, minipage), \parskip = paragraph skip
}
\makeatother


%======================================================
% define new environments
%======================================================
\newtheorem{mydef}{Definition}[section]
\newtheorem{mythm}[mydef]{Theorem}
\newtheorem{beweis}{Beweis}
