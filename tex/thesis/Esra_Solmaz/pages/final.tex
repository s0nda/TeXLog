%
% final.tex
%
%======================================================
\chapter{Abschluss}
%======================================================

Abschließend wird erforscht, ob der Wilsontest als
Primzahltest geeignet bzw. praktisch ist.
Wie es in den Unterkapiteln des zweiten Abschnittes
beschrieben ist, kann man auch anhand
dieses Satzes Primzahlen definieren. Jedoch werden in
den Büchern und Skripten hervorgehoben, dass der
Wilsontest hinsichtlich des Primzahltests ungeeignet ist.

Otto Forster argumentiert in seinem Buch
``Algorithmische Zahlentheorie'', dass obwohl der Satz
von Wilson eine notwendige und hinreichende Bedingung
für die Primalität von $p$ liefert, ist er trotzdem
für praktische Primzahltest ungeeignet, weil es für die
Berechnung von $(p-1)! \mod p$ keinen schnellen
Algorithmus gibt, der mit dem Potenzierungs-Algorithmus
vergleichbar wären (vgl. \cite{forster}, S.57).

Auch Gábor Sas erwähnt in seinem Skriptum, dass dieser
Test sehr ineffizient ist, da wiederum für die Berechnung
der Fakultät keine schnellen Methoden bekannt sind
und müsste $p-2$ Multiplikationen durchführen, um eine
Zahl als Primzahl zu ``entlarven''. Aus diesem Grund
wird der Test praktisch unanwendbar
(vgl. \cite{sasgabor}, S.13).

Als angehende Lehrerin aber werde ich diesen Satz auf
jeden im Unterricht einsetzen, da anhand dieses Satzes
und einer praktischen Tabelle (Kapitel 2.7) die Zahlen
nach der Primalität einfach überprüft werden können.

Allgemein möchte ich andeuten, dass ich diesen Satz
aus dem Grund gewählt habe, da bestimmte Sätze in der
Mathematik für eine (lange) Zeit verloren gegangen
sind und viele Jahre später wiederentdeckt waren.
Auch der Satz von Wilson ist mehr als 700 Jahre später
wiedererfunden worden und somit habe ich mich für
diesen Satz entschieden. Ich wollte mich erkundigen,
warum dieser Satz für mehr als 700 Jahre in Vergessenheit
geraten ist und hinsichtlich der Recherche kam ich auf
den Punkt, dass dieser Satz von Alhazen nicht bewiesen
werden konnte und das könnte ein Anlass für die
Vergessenheit gewesen sein. Außerdem habe ich den Satz
aufgrund der Methodik mit der Fakultät und Kongruenz
sehr interessant gefunden.

Ich bedanke mich beim Herrn Mag. Dr. Andreas Ulovec
für die großartige Unterstützung meiner Bachelorarbeit
und des Weiteren möchte ich in den Vordergrund bringen,
dass ich den Herrn Professor immer weiterempfehlen werde.
