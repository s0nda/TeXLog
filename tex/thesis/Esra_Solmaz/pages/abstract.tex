%
% abstract.tex
%
%{\Huge \textsc{Abstract}}\\
{\LARGE \textbf{ABSTRACT}}\\

Inwiefern ist eine Zahl eine Primzahl? Wie kann ich wissen oder zeigen,
dass sich bei einer Zahl um eine Primzahl handelt? Gibt es dafür eine
Formel, eine Methode, eine Theorie bzw. einen Satz?\\

Im Rahmen dieser Bachelorarbeit beschäftige ich mich mit dem Satz von
Wilson, ein Satz aus der Zahlentheorie, der die Primzahlen
charakterisiertund definiert. Der Satz, wiederentdeckt und benannt nach
John Wilson, lautet kurz und präzise (falls man die Kongruenz einbezieht)
folgendermaßen:

\begin{center}
    \textbf{Ist $\bm{p \in \mathbb{P}}$ eine Prizahl, so ist $\bm{(p-1)! \:\equiv -1 \;(\bmod p)}$}
    
    (Markwig 2008)
\end{center}

Jedoch kann man den Satz auch anders und sogar noch einfacher formulieren.
Diese verschiedenen Formulierungen werde ich in den nächsten Kapiteln und
Abschnitten genauer darstellen.

Schon allein durch das Lesen dieses kurzen Satzes stoßen wir auf die
verschiedenen ``Vokabularen`` und Zeichen der Mathematik. Im ersten Teil
des Satzes ist die Primzahl angegeben. Des Weiteren ist die Fakultät mit
dem Rufzeichen ebenfalls ein Teil des Satzes und im letzten ``Abschnitt''
kommen die Kongruenzen bzw. Restklassen in den Vordergrund. Aus dieser
Tatsache erkennen wir, dass sich der Satz über verschiedene Gebiete der
Mathematik erstreckt und anhand dieses Satzes besteht die Möglichkeit
der Bestimmung der Primzahlen.\\

Die Arbeit besteht aus zwei Teilen. Im ersten Teil der Arbeit stelle ich
die Primzahlen in den Vordergrund, da sie bezüglich des Satzes eine
wichtige Rolle spielen. Dargestellt werden die geschichtlichen Aspekte
und der Begriff der Primzahlen, sowie die Primzahltests.
Im zweiten Abschnitt der Arbeit wird der Satz von Wilson, die Erfinder
und die unterschiedlichen Formulierungen des Satzes, aber auch die
Beweise genau untersucht. Außerdem werden mithilfe des Satzes einige
Zahlen nach der Primalität überprüft.
\newpage
