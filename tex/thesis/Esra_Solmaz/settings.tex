%
% settings.tex
%
%======================================================
% import packages
%======================================================
\usepackage[utf8]{inputenc}%\usepackage[T1]{fontenc}
\usepackage[ngerman]{babel}
\usepackage[autostyle]{csquotes}% load "csquotes" when working with "babel" and "bibtex" to avoid
                                % the warning "'babel/polyglossia' detected but 'csquotes' missing".
\usepackage{amsmath,amsthm,amssymb}
%\usepackage[sc]{mathpazo}% sc = small-cap
\usepackage{mathptmx}% supports Adobe Times Roman (or equivalent) as default text font.
\usepackage{bm}% display bold maths symbols.
\usepackage{float}% used with \begin{figure}[H] to set fixed position.
\usepackage{graphicx}
\usepackage{caption}
\usepackage{colortbl,xcolor}
\usepackage{imakeidx}% create index for later looking-up
\usepackage{ifthen}% conditional statements (if then else)
\usepackage{hyperref}
\usepackage{hypcap}
\usepackage[backend=biber,style=numeric,]{biblatex}% bibliography processor [biber] for BibLatex,
                                                   % [style=alphabetic,]


%======================================================
% configurations
%======================================================
\makeindex % create index for later looking-up
\setlength\parindent{0pt}% set \noindent for entrie file
%
% define variable for language:
%   de : Deutsch
%   en : English
\newcommand{\lang}{en}% default English.
\renewcommand{\lang}{de}% (redundant). Set German as active language.
%
% change the value \proofname (italic "Proof"/"Beweis" by default)
% and its style to bold (\bfseries) and uppercase (\textup).
\addto{\captionsngerman}{%
  \def\proofname{\textbf{\textup{Beweis.}}}
}
%
% command for setting indentation and parskip in minipage.
\setlength{\parskip}{\medskipamount}% set vertical spacing between two paragraphs (glocal), \parskip = paragraph skip
\makeatletter 
\newcommand{\@minipagerestore}{%
  \setlength{\parindent}{0pt}% set indentation at paragraph begin (local)
  \setlength{\parskip}{\medskipamount}% set vertical spacing between two paragraphs (local, minipage), \parskip = paragraph skip
}
\makeatother
%
% define new theorem style
\newtheoremstyle{mytheoremstyle}% name of the style to be used
  {\topsep}% measure of space to leave above the theorem. E.g.: 3pt
  {\topsep}% measure of space to leave below the theorem. E.g.: 3pt
  {\itshape}% name of font to use in the body of the theorem
  {0pt}% measure of space to indent
  {\bfseries}% name of head font
  {. }% punctuation between head and body
  { }% space after theorem head; " " = normal interword space
  {\thmname{\textbf{#1}}\thmnumber{ #2}\thmnote{ (#3)}}
%
%\theoremstyle{mytheoremstyle}

%======================================================
% define new environments
%======================================================
\theoremstyle{mytheoremstyle}% set theorem style
\newtheorem{definition}{Definition}[section]% numbered due to [section]
\newtheorem{theorem}[definition]{Theorem}% numbered due to [defintion]
%
% define new environment for Example. First argument #1
% has default value [Example], that means, the calling
% "\begin{example} ABCD \end{example}" yields
% "Example. ABCD". The whole example-environment is
% within a \par environment (paragraph).
\newenvironment{example}[1][%
  \ifthenelse{% if
    \equal{\lang}{de}
  }{% then
    Beispiel.
  }{% else
    Example.
}]{%
  \par{{\bfseries #1}}
}{\par}
%
